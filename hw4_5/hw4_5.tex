% --------------------------------------------------------------
% This is all preamble stuff that you don't have to worry about.
% Head down to where it says "Start here"
% --------------------------------------------------------------
 
\documentclass[12pt]{article}
 
\usepackage[margin=1in]{geometry} 
\usepackage{amsmath,amsthm,amssymb,algpseudocode,listings}
\usepackage{color}
\usepackage{DejaVuSansMono} 
\usepackage{setspace}
\usepackage{parskip}


\definecolor{Code}{rgb}{0,0,0}
\definecolor{Decorators}{rgb}{0.5,0.5,0.5}
\definecolor{Numbers}{rgb}{0.5,0,0}
\definecolor{MatchingBrackets}{rgb}{0.25,0.5,0.5}
\definecolor{Keywords}{rgb}{0,0,1}
\definecolor{self}{rgb}{0,0,0}
\definecolor{Strings}{rgb}{0,0.63,0}
\definecolor{Comments}{rgb}{0,0.63,1}
\definecolor{Backquotes}{rgb}{0,0,0}
\definecolor{Classname}{rgb}{0,0,0}
\definecolor{FunctionName}{rgb}{0,0,0}
\definecolor{Operators}{rgb}{0,0,0}
\definecolor{Background}{rgb}{0.98,0.98,0.98}

\lstnewenvironment{python}[1][]{
\lstset{
numbers=left,
numberstyle=\ttfamily,
numbersep=1em,
xleftmargin=1em,
framextopmargin=2em,
framexbottommargin=2em,
showspaces=false,
showtabs=false,
showstringspaces=false,
frame=l,
tabsize=4,
% Basic
basicstyle=\ttfamily\small\setstretch{1},
backgroundcolor=\color{Background},
language=Python,
% Comments
commentstyle=\color{Comments}\slshape,
% Strings
stringstyle=\color{Strings},
morecomment=[s][\color{Strings}]{"""}{"""},
morecomment=[s][\color{Strings}]{'''}{'''},
% keywords
morekeywords={import,from,class,def,for,while,if,is,in,elif,else,not,and,or,print,break,continue,return,True,False,None,access,as,,del,except,exec,finally,global,import,lambda,pass,print,raise,try,assert},
keywordstyle={\color{Keywords}\bfseries},
% additional keywords
morekeywords={[2]@invariant},
keywordstyle={[2]\color{Decorators}\slshape},
emph={self},
emphstyle={\color{self}\slshape},
%
}}{}

\newcommand{\N}{\mathbb{N}}
\newcommand{\Z}{\mathbb{Z}}
 
\newenvironment{theorem}[2][Theorem]{\begin{trivlist}
\item[\hskip \labelsep {\bfseries #1}\hskip \labelsep {\bfseries #2.}]}{\end{trivlist}}
\newenvironment{lemma}[2][Lemma]{\begin{trivlist}
\item[\hskip \labelsep {\bfseries #1}\hskip \labelsep {\bfseries #2.}]}{\end{trivlist}}
\newenvironment{exercise}[2][Exercise]{\begin{trivlist}
\item[\hskip \labelsep {\bfseries #1}\hskip \labelsep {\bfseries #2.}]}{\end{trivlist}}
\newenvironment{problem}[2][Problem]{\begin{trivlist}
\item[\hskip \labelsep {\bfseries #1}\hskip \labelsep {\bfseries #2.}]}{\end{trivlist}}
\newenvironment{question}[2][Question]{\begin{trivlist}
\item[\hskip \labelsep {\bfseries #1}\hskip \labelsep {\bfseries #2.}]}{\end{trivlist}}
\newenvironment{corollary}[2][Corollary]{\begin{trivlist}
\item[\hskip \labelsep {\bfseries #1}\hskip \labelsep {\bfseries #2.}]}{\end{trivlist}}


\lstset{basicstyle=\footnotesize\ttfamily,breaklines=true}
\lstset{framextopmargin=50pt,frame=bottomline}

\begin{document}
 
% --------------------------------------------------------------
%                         Start here
% --------------------------------------------------------------
 
\title{Homework 4 \& 5}%replace X with the appropriate number
\author{Jeremy Wright\\ %replace with your name
CSE598 - Analysis of Algorithms} %if necessary, replace with your course title
 
\maketitle
\begin{problem}{1}
    Show how to multiple the complex numbers $a+bj$ and$c+dj$ using only three
    real multiplications\dots

    Given: 
        \begin{align*}
        R+jI &= (a+jb)\cdot(c+jd) \\
        &= (ac - bd) + j(bc + ad)
    \end{align*}

    A common pattern in Digital Signal Processing is to break out the operations
    into 3 constants.
    \begin{align*}
        \label{eq:components}
        k_1 &= a(c + d) &= ac + ad \\
        k_2 &= d(a + b) &= ad + bd \\
        k_3 &= c(b - a) &= bc - ac \\
    \end{align*}
    From these we can define the Real part as
    \begin{align*}
        \Re &= k_1 - k_2 \\
            &= ac + ad - (ad + bd) \\
            &= ac + ad - ad - bd \\
            &= ac - bd \\
    \end{align*}
    The Imaginary Part as
    \begin{align*}
        \Im &= k_1 + k_3            \\
            &= ac + ad + bc - ac    \\
            &= ab + bc
    \end{align*}

    In Python we can clearly see this uses only 3 multiplications.
    \begin{python}
        def complex_mul(a, b, c, d):
            k1 = a(c + d)
            k2 = d(a + b)
            k3 = c(b - a)
            return (k1 - k2) + (k1 - k3)*j
    \end{python}
\end{problem}

\begin{problem}{2}
Consider in the closest pair of points problem, splitting the points along
$ \frac{n}{4}$ and $\frac{3n}{4}$. Is the complexity still $O(n lg n)$? Yes

Proof:

Consider the base case $ n >= 4$:
\begin{equation*}
    T(n) = T(\frac{n}{4}) + T(\frac{3n}{4}) + n
\end{equation*}
Substitute $n lg n$
\begin{align*}
    T(n) =& T(\frac{n}{4}) + T(\frac{3n}{4}) + n\\
    T(n) \le& c \frac{n}{4} \log_2{\frac{n}{4}}    + c \frac{3n}{4}\log_2{\frac{3n}{4}} + cn\\
    \le& c \left( \frac{n}{4} \log_2{n} - \frac{n}{4} \log_2{4} \right)
    + c \left(\frac{3n}{4} \log_2{3n} - \frac{3n}{4} \log_2{4} \right) + cn \\
    \le& c \left( \frac{n}{4} \log_2{n} - \frac{2n}{4} \right)  + c \left( \frac{3n}{4} \log_2{3n} - \frac{3n\cdot2}{4} \right) + cn \\
    \le& c\left(\frac{n}{4} \log_2{n} - \frac{1}{2}n\right)   + c \left(\frac{3n}{4} \log_2{3} + \frac{3n}{4} \log_2{n} - \frac{3n}{2}\right) + cn\\
    \le& c\frac{n}{4} \log_2{n} - \frac{1}{2}cn + \frac{3nc}{4} \log_2{n} - cn
    + cn\\
    \le& c\frac{n}{4} \log_2{n} + \frac{3cn}{4}\log_2{n} \\
    \le& cn\log_2{n}
\end{align*}
Intuitively this makes sense because we are still dividing the space into some
log relatable ratio for each subproblem. 

Consider dividing the space into $\sqrt{n}$ and $n-\sqrt{n}$. This problem will
not remain at $n \log_2{n}$ since the division of the space is not an even
recurrence. Instead there is a linear adjustment in the form of $n - \dots$
\begin{align*}
    T(n) &= T(n^{\frac{1}{2}}) + T(n-n^{\frac{1}{2}}) + nc\\
    \le& n^{\frac{1}{2}} \log_2{n^{\frac{1}{2}}} + \left(n-
    n^{\frac{1}{2}}\right) \log_2\left(n - n^{\frac{1}{2}}\right) c + cn
    \dots
\end{align*}
Eventually we get to where we cannot resolve the logs with the roots.  So let's
try a greater complexity: $n^2$
\begin{align*}
    \le& \left( n^{\frac{1}{2}} \right)^2 + \left( n - n^\frac{1}{2} \right)^2
    + nc\\
    \le& n + n^2 - n + nc\\
    \le& n^2 + n(1 - 1 + 1)c\\
    \le& n^2 + nc\\
    \le& \lim_{n\to \infty}\left(n^2 + nc\right)\\
    \le& cn^2
\end{align*}
Thus by adding an irregular division to the sub problems we increase the
complexity to recombine those subproblems.
\end{problem}



\end{document}
