% --------------------------------------------------------------
% This is all preamble stuff that you don't have to worry about.
% Head down to where it says "Start here"
% --------------------------------------------------------------
 
\documentclass[12pt]{article}
 
\usepackage[margin=1in]{geometry} 
\usepackage{amsmath,amsthm,amssymb,algpseudocode,listings}
 
\newcommand{\N}{\mathbb{N}}
\newcommand{\Z}{\mathbb{Z}}
 
\newenvironment{theorem}[2][Theorem]{\begin{trivlist}
\item[\hskip \labelsep {\bfseries #1}\hskip \labelsep {\bfseries #2.}]}{\end{trivlist}}
\newenvironment{lemma}[2][Lemma]{\begin{trivlist}
\item[\hskip \labelsep {\bfseries #1}\hskip \labelsep {\bfseries #2.}]}{\end{trivlist}}
\newenvironment{exercise}[2][Exercise]{\begin{trivlist}
\item[\hskip \labelsep {\bfseries #1}\hskip \labelsep {\bfseries #2.}]}{\end{trivlist}}
\newenvironment{problem}[2][Problem]{\begin{trivlist}
\item[\hskip \labelsep {\bfseries #1}\hskip \labelsep {\bfseries #2.}]}{\end{trivlist}}
\newenvironment{question}[2][Question]{\begin{trivlist}
\item[\hskip \labelsep {\bfseries #1}\hskip \labelsep {\bfseries #2.}]}{\end{trivlist}}
\newenvironment{corollary}[2][Corollary]{\begin{trivlist}
\item[\hskip \labelsep {\bfseries #1}\hskip \labelsep {\bfseries #2.}]}{\end{trivlist}}
 
\begin{document}
 
% --------------------------------------------------------------
%                         Start here
% --------------------------------------------------------------
 
\title{Homework 3}%replace X with the appropriate number
\author{Jeremy Wright\\ %replace with your name
CSE598 - Analysis of Algorithms} %if necessary, replace with your course title
 
\maketitle
\begin{problem}{1}
    The following table demonstrates the cost over time:
\\
    \begin{tabular}{|c | c | c | c |}\hline
        Power & Step & Cost & Time-Value\\ \hline
        $0$   & 0    & 1    & $1 + lg(0) = 1$ \\ \hline
        $2^0$ & 1    & 1    & $1 + lg(0) = 1$ \\ \hline
        $2^1$ & 2    & 2    & $1 + lg(0) = 2$ \\ \hline
            & 3    & 1    & $1 + lg(0) = 2.58 bank has 3.58$ \\ \hline
        $2^2$ & 4    & 4    & $ bank has .58$ \\ \hline
            & 5    & 1    & $ bank has 2.32 + 0.58 = 2.91$ \\ \hline
            & 6    & 1    &  \\ \hline
            & 7    & 1    &  \\ \hline
        $2^3$ & 8    & 8    &  \\ \hline
            & 9    & 1    &  \\ \hline
            & 10   & 1    &  \\ \hline
            & 11   & 1    &  \\ \hline
            & 12   & 1    &  \\ \hline
            & 13   & 1    &  \\ \hline
            & 14   & 1    &  \\ \hline
            & 15   & 1    &  \\ \hline
        $2^4$ & 16   & 16    & $1 + lg(0) = 1$ \\ \hline
    \end{tabular}

    Equations then follow:
    \begin{equation}
        \displaystyle\sum\limits_{n=2^2+1}^{2^3} lg(n) = 28.98
    \end{equation}
    \begin{equation}
       \label{eq:twofour}
        \displaystyle\sum\limits_{n=2^4+1}^{2^5} lg(n) = 73.41
    \end{equation}
    After \ref{eq:twofour}, the bank has 32 left over. Thus lg is too much, can we
    do constant?
    \begin{equation}
       \label{eq:const_3}
        \displaystyle\sum\limits_{n=2^3+1}^{2^4} 2 = 16
    \end{equation}
    \begin{equation}
       \label{eq:const_4}
        \displaystyle\sum\limits_{n=2^4+1}^{2^5} 2 = 32
    \end{equation}
    Equation \ref{eq:const_3} covers the previous costs, and \ref{eq:const_4} breaks
even between $2^4$ and $2^5$ $\therefore$ this algorithm performs constant time
amortized $O(3)$ 1 for the actual operation, 2 for the bank.
\end{problem}
\begin{problem}{2}

\end{problem}
\begin{problem}{3}
    Base case: if the tree has 1 element:
    \begin{equation}
        size(x) = 1 \rightarrow 2^{rank(x)} = 2^0 = 1
    \end{equation}

    Inductive Case:
    Path Compression assures that the unions assumes that the rooted rank will be
    greater than the rank will be greater than the rank of the appended subtree.
    Therefore the root of a set will be strictly greater than it's subtrees
    after the union operation.

    When union runs there are 2 cases for the two separate sets. $set_1$ rank
    = $set_2$ rank.

    Size after the union 
    \begin{equation}
        size_2'.size = size(s2) + size(s1)
    \end{equation}
    Size after the ranks are equal the root set is chosen arbitrarily.
    \begin{equation}
        s_2'.size = s_2.rank + 1
    \end{equation}
    \begin{equation}
        s_2'.size \ge 2^{S_1.rank} + 2^{S_2.rank}
    \end{equation}
\end{problem}

\begin{problem}{4} 
a)\\
\begin{itemize}
    \item Run Depth first search to the bottom of the tree.
    \item Then union all recursive sets together.
    \item once all ancestors have been visited - it colors the nodes black.
    \item once the deepest ancestor of u \& v then will be black $\rightarrow$
        print line 10.
\end{itemize}

b)\\
The recursive call creates separate make sets on entering the recursive call.
These sets are not unioned until the recursion reaches the bottom. Only then
do the sets begin to union. This at each recurse the depth of recursion matches
the number of sets matches the depth of the node in T

c)\\
Assume that LXCE does not print the least common ancestor. LXCE starts at a node
and performs depth first search to the deepest leaf. This leaf is then colored
black. LXCE then iterates over the companion node of the \{u,v\} pair and prints
out the colored black node. If LCA does not find the LCA, then the depth first
search does not find the deepest node $\rightarrow$ a contradiction.

d)\\
\begin{itemize}
    \item makeset $O(1)$.
    \item find(v) w\ path compression $\rightarrow lg* n$
    \item for each child in T do
        \begin{itemize}
            \item LCA(v)
            \item Union $\rightarrow u \cdot lg* n$
        \end{itemize}
\end{itemize}
    Run time is dominated by running Find on each node u, hence $O(u \cdot lg*
    v)$



\end{problem}
\begin{problem}{5}
    $Join(T_1, T_2)$ if every element in $T_1$ is less that $T_2$, then the join
    of $T_1$ to $T_2$ is simply the right rotation of $T_2$ with $T_1$ tacked to
    the left of $T_2$'s root.

    $Find(a)$ remembering if a is the right or left child of it's parent.
    Extract a's subtree as a new tree and rotate right if a was a right-child,
    and rotate left if a was a left-child. 
\end{problem}
 
\end{document}
