% --------------------------------------------------------------
% This is all preamble stuff that you don't have to worry about.
% Head down to where it says "Start here"
% --------------------------------------------------------------
 
\documentclass[12pt]{article}
 
\usepackage[margin=1in]{geometry} 
\usepackage{amsmath,amsthm,amssymb,algpseudocode,listings}
 
\newcommand{\N}{\mathbb{N}}
\newcommand{\Z}{\mathbb{Z}}
 
\newenvironment{theorem}[2][Theorem]{\begin{trivlist}
\item[\hskip \labelsep {\bfseries #1}\hskip \labelsep {\bfseries #2.}]}{\end{trivlist}}
\newenvironment{lemma}[2][Lemma]{\begin{trivlist}
\item[\hskip \labelsep {\bfseries #1}\hskip \labelsep {\bfseries #2.}]}{\end{trivlist}}
\newenvironment{exercise}[2][Exercise]{\begin{trivlist}
\item[\hskip \labelsep {\bfseries #1}\hskip \labelsep {\bfseries #2.}]}{\end{trivlist}}
\newenvironment{problem}[2][Problem]{\begin{trivlist}
\item[\hskip \labelsep {\bfseries #1}\hskip \labelsep {\bfseries #2.}]}{\end{trivlist}}
\newenvironment{question}[2][Question]{\begin{trivlist}
\item[\hskip \labelsep {\bfseries #1}\hskip \labelsep {\bfseries #2.}]}{\end{trivlist}}
\newenvironment{corollary}[2][Corollary]{\begin{trivlist}
\item[\hskip \labelsep {\bfseries #1}\hskip \labelsep {\bfseries #2.}]}{\end{trivlist}}
 
\begin{document}
 
% --------------------------------------------------------------
%                         Start here
% --------------------------------------------------------------
 
\title{Homework 1}%replace X with the appropriate number
\author{Jeremy Wright\\ %replace with your name
CSE598 - Analysis of Algorithms} %if necessary, replace with your course title
 
\maketitle
\begin{problem}{1.4}
    One must iterate through all the students. Since there are more students
    than hospitals, every hospital must be offered every student in order to
    prevents types of instability.

    \begin{proof}
        Consider:\\
        \begin{algorithmic}
            \ForAll{h in hs}
            \State $h \gets top-student$
            \EndFor
        \end{algorithmic}
        In this case there is at least a student that wasn't offered to
        a hospital, thus one must iterate over the students.
    \end{proof}
    \lstset{language=Python}
    \begin{lstlisting}
assignments = []
while(a student is unassigned \& not offered by all hospitals):
    Choose Student s
    Let h be the highest ranking hospital in s preference list to whom h has not offered
    if(h.has_available_slot()):
        assignments.append(s.assign(h))
    else:
        s` = h.assigned()
        if(h.prefers(s`)):
            pass #S remains unassigned
        else:
            assignments.append(s.assign(h))
            assignments.remove(s`) #s` becomes free
#assignments now describes the pair of assigned hospitals and students
    \end{lstlisting}

    Hospitals are offered every student, and will only trade p thus, the first
    type of instability cannot occur since the only way a student s' could be
    unemployed is if he was assigned to a hospital and a hospital fired him in
    favor of a better candidate

    The second type of instability can only occur where s' prefers a hospital h,
    and h prefers s'. If h is never offered s'  since if it was offered and
    h preferred it, s' would be assigned h. Since a hospital sees every student this is a contradiction
\end{problem}
\begin{problem}{4.3}
    To show that we are minimizing the number of trucks without sending boxes
    out or order, we will show that filling each truck to full allows the
    shipping to stay ahead of the box stream. 

    We want to minimize the time spend in the loading depot, so we can maximize
    the number of trucks actually delivering product. Thus we define the loading
    time $L_t$. We want to assure that each successive loading time is less that
    the previous time. Thus $L_t_j \geq L_t_i$.  By filling each truck to
    maximum we assure each successive truck will have the same or equal number
    of boxes. Thus we stay ahead of the box stream. 
    
    If we allowed boxes to arrive out of order, we could optimize this by
    filling the volume of the truck more efficiently. This would further
    maximize each loading time, and allow one to send fewer truck.
\end{problem}
\begin{problem}{4.13} 
In order to maximize the customer satisfaction a new metric must be used which
combines the run time ($t$) with the customer weight ($w$).
\begin{equation}
\end{equation}
The place each $job_i$ in a minimum priority queue with their respective priority
$z_i$.
\begin{algorithmic}
    \Require{$z_i = t_i \cdot w_i$}
    \Require{Place z in a min priority queue zs}
    \ForAll{job in zs}
        \State $Schedule \gets job$
    \EndFor
\end{algorithmic}

By combining the weight and time into a new metric, jobs which are short, but
have a high customer weight will be scheduled before jobs which are longer but
have a lesser customer weight.
\end{problem}
 
\begin{problem}{4.16}
    This problem is similar to the interval scheduling problem except that the
    transaction n must be placed into a room or ``interval''. We must try to put
    as many transactions into as many intervals as possible.
    \begin{algorithmic}
        \Require $xs$ be the list of times bounding a window, $x_L$ for the Low time, and
        $x_H$ for the high time. 
        \Require $ns$ be the list of transactions 
        \ForAll{n in ns}
            \ForAll{x in xs}
            \If {$x_L < n < x_H$}
                    \State $n \gets x$ \Comment{transaction n is marked to occur
                    at time x}
                \EndIf
            \EndFor
        \EndFor
\end{algorithmic}
Let $n\in \Z$.  Then yada yada.
\end{problem}
 
\begin{proof}
Blah, blah, blah.  I'm so smart.
\end{proof}
 
% --------------------------------------------------------------
%     You don't have to mess with anything below this line.
% --------------------------------------------------------------
 
\end{document}
